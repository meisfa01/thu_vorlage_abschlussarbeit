
% ##################################################
% Seitenlayout
% ##################################################
\usepackage[automark,headsepline]{scrlayer-scrpage}


\pagestyle{scrheadings}
\raggedbottom

\newcommand{\publish}{}			%auskommentiern f�r gro�en Seitenrand
\ifdefined\publish
\usepackage[bindingoffset=0mm,
left=20mm,
right=20mm,
bottom=10mm,
top=5mm,
footskip=15mm,
headheight=15mm,
includehead, includefoot
] {geometry} 
\else
\usepackage[bindingoffset=0mm,
left=5mm,
right=9cm,
bottom=5mm,
top=5mm,
paperwidth=27cm,
paperheight=25cm,
marginparwidth=8cm,
twoside=false,
% ncludehead, includefoot
] {geometry} 
\fi

% ##################################################
% Sprache
% ##################################################
%\usepackage[utf8]{inputenc}  % Eingabe von Umlauten ermöglichen
%\usepackage[latin1]{inputenc}  % Eingabe von Umlauten ermöglichen
\usepackage[T1]{fontenc}                     
\usepackage{textcomp}
\usepackage[ngerman]{babel} % Sprachpaket
\usepackage[babel,german=quotes]{csquotes}
%\addto\extrasenglish{\languageshorthands{ngerman}}          
%\usepackage{hyphenat}                   
\selectlanguage{ngerman}


% ##################################################
% Bibliografie
% ##################################################
\immediate\write18{if exist copyBib.bat start /B copyBib.bat}
\usepackage[sorting=nty,
%                                       firstinits=true,
giveninits=true,
style=numeric,
backend=bibtex,                           % bibtex8 biber,
isbn=false,
doi=false,
sortcites=true,
maxnames=4]{biblatex}
%\addbibresource{C:/Dokumente/Fachliteratur/Bibliothek}
\addbibresource{../../../Fachliteratur/Bibliothek}								%eigene Ressource angeben

\newcommand{\MiniBibliography}{\vspace{1em}%
	{\renewcommand{\baselinestretch}{1.25}%
		\let\emph\relax \DeclareTextFontCommand{\emph}{\normalfont}         % emph -> normal
		\renewcommand{\chapter}[2]{\subsection*{##2}}%
		\printbibliography}}



% ##################################################
% PGF Plots
% ##################################################
\usepackage{pgfplots}
\usepackage{circuitikz}[=v1.2.7] %[siunitx]
\pgfplotsset{compat=1.7}
%\pgfplotsset{compat=1.15}
%\pgfplotsset{compat=1.5}
\usepgfplotslibrary{patchplots}
\usepgfplotslibrary{fillbetween}
\usepackage{pgfplotstable}
\pgfmathsetmacro{\PI}{3.141592654}

\usepackage{tikz}
\usepackage{mathtools}
\usetikzlibrary{arrows}
\usetikzlibrary{calc}
\usetikzlibrary{positioning}        % verkürzte Angabe of
\usetikzlibrary{fit}
\usetikzlibrary{arrows.meta}
\usetikzlibrary{circuits.logic.IEC}
\usetikzlibrary{circuits.plc.ladder} %
%\usetikzlibrary{calligraphy}
% Alle Tikz-Grafiken einzeln als pdf und svg ablegen
\ifdefined\externalizeTikz
\usetikzlibrary{external}
%\immediate\write18{if not exist Abb md Abb}
\tikzset{external/system call={
		%                                       echo \texsource %&& %
		pdflatex \tikzexternalcheckshellescape --extra-mem-bot=1000000 --extra-mem-top=100000 -halt-on-error -interaction=batchmode -jobname "\image" "\texsource" %
		%pdftops -eps "\image".pdf  &&
		&& inkscape "\image".pdf --export-type="pdf,svg,emf"
		%--export-background=#ffffff --export-background-opacity=1
		%%                                    && convert \image.pdf \image.png
		%           %                         &&        pdf2png "\image" %
		%%                                    && cd Abb
		%%                                    && del "\image".dpth % Abb/"\image".log Abb/"\image".md5 Abb/"\image".*xml %
	}  %
} %
\tikzexternalize[prefix=Abb/, shell escape=-enable-write18] %
\fi

\def\blockwidth{1.6}
\tikzset{
	Block/.style args={#1}{%
		rectangle, draw, inner sep=0mm, fit={(0,0) (\blockwidth,1)},
		append after command={\pgfextra{\let\mainnode=\tikzlastnode %
				\node[anchor=center, inner sep=0] at (\mainnode.center) {{#1}};%
			}
		}
	},
	PGlied/.style args={#1}{%
		rectangle, draw, inner sep=0mm, fit={(0,0) (\blockwidth,1)},
		append after command={\pgfextra{\let\mainnode=\tikzlastnode %
				\node[above=0.25em,anchor=base west, inner sep=0] at (\mainnode.north west) {{#1}};%
				\begin{scope}[xscale=\blockwidth,shift={($(\mainnode.south west)$)}]
					\draw[very thick] (0,0.8) -- (1,0.8); %
				\end{scope} %
			}
		}
	},
	IGlied/.style args={#1}{%
		rectangle, draw, inner sep=0mm, fit={(0,0) (\blockwidth,1)}, %
		append after command={\pgfextra{\let\mainnode=\tikzlastnode %
				\node[above=0.25em,anchor=base west, inner sep=0] at (\mainnode.north west) {{#1}};%
				\begin{scope}[xscale=\blockwidth,shift={($(\mainnode.south west)$)}]
					\draw[very thick] (0.01,0.01) -- (0.99,0.99); %
				\end{scope} %
			}
		}
	},
	DGlied/.style args={#1}{%
		rectangle, draw, inner sep=0mm, fit={(0,0) (\blockwidth,1)}, %
		append after command={\pgfextra{\let\mainnode=\tikzlastnode %
				\node[above=0.25em,anchor=base west, inner sep=0] at (\mainnode.north west) {{#1}};%
				\begin{scope}[xscale=\blockwidth,shift={($(\mainnode.south west)$)}]
					\draw[very thick] (0,0.1) -- (0.1,0.1) -- (0.1,1) -- (0.1,0.1) -- (1,0.1); %
				\end{scope} %
			}
		}
	},
	PT1/.style args={#1#2}{%
		rectangle, draw, inner sep=0mm, fit={(0,0) (\blockwidth,1)}, %
		append after command={\pgfextra{\let\mainnode=\tikzlastnode %
				\node[above=0.25em,anchor=base west, inner sep=0] at (\mainnode.north west) {{#1}};%
				\node[above=0.25em,anchor=base east, inner sep=0] at (\mainnode.north east) {{#2}};%
				\begin{scope}[xscale=\blockwidth,shift={($(\mainnode.south west)$)}]
					\draw[very thick] (0.01,0) .. controls (0.2,1) and (0.4,0.95) .. (1,0.95); %
				\end{scope} %
			}
		}
	},
	PT2/.style args={#1#2}{%
		rectangle, draw, inner sep=0mm, fit={(0,0) (\blockwidth,1)}, %
		append after command={\pgfextra{\let\mainnode=\tikzlastnode %
				\node[above=0.25em,anchor=base west, inner sep=0] at (\mainnode.north west) {{#1}};%
				\node[above=0.25em,anchor=base east, inner sep=0] at (\mainnode.north east) {{#2}};%
				\begin{scope}[xscale=\blockwidth,shift={($(\mainnode.south west)$)}]
					\draw[very thick] (0.01,0)              .. controls (0.1,0.8)         and (0.25,1)       .. (0.35,0.8)
					.. controls (0.45,0.65)    and (0.55,0.75)         .. (0.6,0.8)
					.. controls (0.7,0.87)      and (0.75,0.82)         .. (0.8,0.8)
					.. controls (0.85,0.78)    and (0.9,0.78)           .. (1,0.8) ;
					-- (1, 0.8); %
				\end{scope} %
			}
		}
	},
	Totzeit/.style args={#1}{%
		rectangle, draw, inner sep=0mm, fit={(0,0) (\blockwidth,1)}, %
		append after command={\pgfextra{\let\mainnode=\tikzlastnode %
				\node[above=0.25em,anchor=base west, inner sep=0] at (\mainnode.north west) {{#1}};%
				%                                                     \draw[xscale=\blockwidth,very thick, inner sep=0]  (0,0.01) -- (0.2,0.01) -- (0.2,0.8) -- (1,0.8);
				\begin{scope}[xscale=\blockwidth,shift={($(\mainnode.south west)$)}]
					\draw[very thick] (0,0.01) -- (0.2,0.01) -- (0.2,0.8) -- (1,0.8);
				\end{scope} %
			}
		}
	},
	Kennlinie/.style args={#1}{%
		rectangle, draw, inner sep=0mm, fit={(0,0) (\blockwidth,1)},
		append after command={\pgfextra{\let\mainnode=\tikzlastnode %
				%                                                     \draw (0.05,0.05) rectangle (\blockwidth-0.05,0.95);
				\node[anchor=center, inner sep=0] at (\mainnode.center) {{#1}};%
				\begin{scope}[shift={($(\mainnode.south west)$)}]
					\draw (0.05,0.05) rectangle (\blockwidth-0.05,0.95);
				\end{scope} %  
			}
		}
	},
	Saturation/.style args={#1}{%
		rectangle, draw, inner sep=0mm, fit={(0,0) (\blockwidth,1)},
		append after command={\pgfextra{\let\mainnode=\tikzlastnode %
				%                                                     \draw[xscale=\blockwidth, thin] (0.2,0.5) -- (0.8,0.5);
				%                                                     \draw[xscale=\blockwidth, thin] (0.5,0.1) -- (0.5,0.9);
				%                                                     \draw[xscale=\blockwidth, very thick] (0.2,0.2) -- (0.35,0.2) -- (0.65,0.8) -- (0.8,0.8);
				\begin{scope}[xscale=\blockwidth,shift={($(\mainnode.south west)$)}]
					\draw[very thick] (0.2,0.2) -- (0.35,0.2) -- (0.65,0.8) -- (0.8,0.8);
					\draw[thin]  (0.2,0.5) -- (0.8,0.5) (0.5,0.1) -- (0.5,0.9);
				\end{scope} %
			}
		}
	},
	ReLU/.style args={#1}{%
		rectangle, draw, inner sep=0mm, fit={(0,0) (\blockwidth,1)},
		append after command={\pgfextra{\let\mainnode=\tikzlastnode %
				\begin{scope}[xscale=\blockwidth,shift={($(\mainnode.south west)$)}]
					\draw[very thick] (0.2,0.3) -- (0.5,0.3) -- (0.8,0.8);
					\draw[thin]  (0.2,0.3) -- (0.8,0.3) (0.5,0.1) -- (0.5,0.9);
				\end{scope} %
			}
		}
	},
	DeadZone/.style args={#1}{%
		rectangle, draw, inner sep=0mm, fit={(0,0) (\blockwidth,1)},
		append after command={\pgfextra{\let\mainnode=\tikzlastnode %
				\begin{scope}[xscale=\blockwidth,shift={($(\mainnode.south west)$)}]
					\draw[very thick] (0.2,0.15) -- (0.4,0.5) -- (0.6,0.5) -- (0.8,0.85);
					\draw[thin]  (0.2,0.5) -- (0.8,0.5) (0.5,0.1) -- (0.5,0.9);
				\end{scope} %
			}
		}
	},
	Hysterese/.style args={#1}{%
		rectangle, draw, inner sep=0mm, fit={(0,0) (\blockwidth,1)},
		append after command={\pgfextra{\let\mainnode=\tikzlastnode %
				\begin{scope}[xscale=\blockwidth,shift={($(\mainnode.south west)$)}]
					\draw[thin]  (0.2,0.5) -- (0.8,0.5) (0.5,0.1) -- (0.5,0.9);
					\draw[-<, very thick] (0.1,0.15)  .. controls (0.3,0.15)  .. (0.4,0.5) -- (0.45,0.7);
					\draw[very thick] (0.4,0.5) .. controls (0.5, 0.85) .. (0.9,0.85);
					\draw[very thick] (0.1,0.15)  .. controls (0.5,0.15) .. (0.6,0.5);
					\draw[>-,very thick] (0.55,0.3) -- (0.6,0.5) .. controls (0.7, 0.85) .. (0.9,0.85);
				\end{scope} %
			}
		}
	},
	RateLimiter/.style args={#1}{%
		rectangle, draw, inner sep=0mm, fit={(0,0) (\blockwidth,1)},
		append after command={\pgfextra{\let\mainnode=\tikzlastnode %
				\begin{scope}[xscale=\blockwidth,shift={($(\mainnode.south west)$)}]
					\draw[thin]  (0.2,0.5) -- (0.8,0.5) (0.5,0.1) -- (0.5,0.9);
					\draw[very thin]  (0.25,0.15) -- (0.75,0.85);
					\draw[very thick] (0.27,0.35) arc (-60:-35:0.75 and 0.6) -- (0.73,0.82);
					\draw[thin] (0.27,0.35) arc (-60:-2:0.75 and 0.6);
				\end{scope} %                               }
		}
	}
}
}

\tikzstyle{prod} = [draw, circle, minimum size=7mm, %label={[anchor=center]$\times$},
append after command={\pgfextra{\let\mainnode=\tikzlastnode %
	\node[anchor=center, inner sep=0] at (\mainnode.center) {\large $\times$};%
}
}]



\tikzstyle{sum} = [draw, circle, inner sep=2mm]
\tikzstyle{box} = [draw, rectangle, inner sep=2mm]
\tikzstyle{junction} = [circle, fill, inner sep=0.5mm]

%\def\subnode#1#2{\tikz[remember picture,baseline=(#1.base),inner sep=0pt,outer sep=0pt,minimum width=0pt]\node(#1){#2};}

\pgfplotscreateplotcyclelist{THUcolors}{
%blue!31!green,every mark/.append style={fill=blue!10!black},mark=+
{THUBlue},
{THUGreen},
{THUDarkBlue},
}


\pgfplotsset{
every axis/.append style={
	cycle list name=THUcolors,
},
AmplGang/.style={
	xmode=log,
	enlarge x limits=0,
	grid=both,
	y filter/.code={\pgfmathparse{20*log10(\pgfmathresult)}},           %dB Skala
	ytick distance=20,
	scale only axis,
	no markers,
	xlabel={$\omega$},
	ylabel={$|G(j\omega)|$\ in dB}, ylabel near ticks, y label style={at={(rel axis cs:0,1)},rotate=-90,anchor=south east,inner sep=1mm},
	every axis plot/.append style={line width=2pt},
	width=6cm, height=3cm
},
AmplGang/.belongs to family=/pgfplots/scale,
PhasenGang/.style={
	xmode=log,
	enlarge x limits=0,
	grid=both,
	ytick distance=45,
	scale only axis,
	no markers,
	xlabel={$\omega$},
	ylabel={$\angle G(j\omega)$\ in\ $^\circ$}, ylabel near ticks, y label style={at={(rel axis cs:0,1)},rotate=-90,anchor=south east,inner sep=1mm},
	every axis plot/.append style={line width=2pt},
	%                         cycle list name=THUcolors,
	width=6cm, height=3cm
},
PhasenGang/.belongs to family=/pgfplots/scale,
Ortskurve/.style={
	%                         enlarge x limits=1,
	grid=none,
	%                         no markers,
	%                         ymajorticks=false, xmajorticks=false,
	scale only axis,
	scale mode=scale uniformly,
	xlabel={Re},
	ylabel={Im},
	axis lines=middle,
	every axis plot/.append style={line width=2pt},
	before end axis/.code={
		\addplot [domain=-180:180, samples=100, color=black, thin] ({cos(x)},{sin(x)});
	},
	height=7cm,
},
OrtskurveOhneKreis/.style={
	grid=none,
	scale only axis,
	scale mode=scale uniformly,
	xlabel={Re},
	ylabel={Im},
	axis lines=middle,
	every axis plot/.append style={line width=2pt},
	height=7cm,
},
SignalKont/.style={
	%                         enlarge x limits=1,
	grid=both,
	no markers,
	%                         ymajorticks=false, xmajorticks=false,
	scale only axis,
	extra x tick style={grid=none},
	xlabel={$t$}, x label style={at={(axis cs:1,0)},anchor=north},
	ylabel near ticks, y label style={at={(axis cs:0,1)},rotate=-90,anchor=south east},
	axis lines=middle,
	every axis plot/.append style={line width=2pt},
	cycle list name=THUcolors,
},
}

\usetikzlibrary{patterns,decorations.pathmorphing,decorations.markings}

\tikzstyle{spring}=[thick,decorate,decoration={zigzag,pre length=0.4cm,post length=0.4cm,segment length=5}]
\tikzstyle{damper}=[thick,decoration={markings, 
mark connection node=dmp,
mark=at position 0.5 with
{
	\node (dmp) [thick,inner sep=0pt,transform shape,rotate=-90,minimum width=15pt,minimum height=3pt,draw=none] {};
	\draw [thick] ($(dmp.north east)+(2pt,0)$) -- (dmp.south east) -- (dmp.south west) -- ($(dmp.north west)+(2pt,0)$);
	\draw [thick] ($(dmp.north)+(0,-5pt)$) -- ($(dmp.north)+(0,5pt)$);
}
}, decorate]
\tikzstyle{ground}=[fill,pattern=north east lines, pattern color=black,draw=none]



%\tikzset{>={Latex[width=1.7mm,length=2.5mm]}}
\tikzset{>={Latex[width=2mm,length=3mm]}}



% ##################################################
% Packages
% ##################################################
%\usepackage[draft]{graphicx} % Paket zum Einbinden von Graphiken und Bildern, möglich [dvips]
\usepackage[\ifdefined\draft draft=true\fi]{grffile}         % vgl. https://tex.stackexchange.com/questions/110513/unknown-graphics-extension-1-png\\
\graphicspath{{./pictures/}}        %default path for pictures
%\usepackage{svg}
%\usepackage{pdfpages} % Einbinden von externen pdf-Dateien



\usepackage[shortlabels]{enumitem}
\setlist{nosep} % or \setlist{noitemsep} to leave space around whole list

% Mathe
\usepackage{amsmath}                             % Paket f�r Mathematische Formeln ([fleqn] bereits als doc option)
\usepackage{nicefrac}                               % Sch�ne Br�che
\usepackage{xfrac}                                    % Diagonale Br�che \sfrac

% Symbole
\usepackage{amssymb}                            % Paket f�r Mathematische Symbole
\usepackage{stmaryrd}                             % Blitz-Symbol \lightning
\usepackage{trfsigns}                               % Laplace-Symbol
\usepackage{mathcomp}                          % Paket mit Gradzeichen
\usepackage[official]{eurosym} 				 % Euro-Symbol

%\usepackage{units}                                 % Paket f�r units-Darstellung im Text
%\usepackage[exponent-product={\cdot},retain-explicit-plus]{siunitx}

% Formatierung
\interfootnotelinepenalty=10000           % Umbr�che von Fu�noten verbieten
\usepackage{longtable}
\usepackage{tabularray}
\usepackage{tabularx}

% Zeitplanung
\usepackage{pgfgantt}
\usepackage{datetime}
\renewcommand{\dateseparator}{.} % Datumsformat anpassen

\newcounter{myweek}
\setcounter{myweek}{40} % Startwoche (Kalenderwoche 40 f�r 1. Oktober 2024)

\newcommand{\ganttweektitle}{%
	\gantttitle{KW \themyweek}{7}%
	\stepcounter{myweek}%
}

% Layout
\usepackage{tocbasic}
\renewcommand{\baselinestretch}{1.5} % 1,5-facher Zeilenabstand

% Links
\usepackage{xcolor}             % for colored text
\definecolor{codegreen}{rgb}{0,0.6,0}
\definecolor{codegray}{rgb}{0.5,0.5,0.5}
\definecolor{codepurple}{rgb}{0.58,0,0.82}
\definecolor{backcolour}{rgb}{1,1,1}
\definecolor{peuscherblau}{rgb}{0,0.33,0.63}
\usepackage[colorlinks=true]{hyperref}
\hypersetup{
	linkcolor=peuscherblau,
	urlcolor=peuscherblau}