\newpage
\thispagestyle{plain}

\chapter*{Notation}
\addcontentsline{toc}{chapter}{Notation}

Alle mathematischen Ausdrücke dieser Arbeit sind in der nachfolgend beschriebenen Notation dargestellt:\\
\begin{table}[ht]
	\begin{tabular}{lll}
		Skalare:    & kursiv & $u(t)$ \\
		Vektoren: & fett & $\mathbf{x}(t)$ \\
		Matrizen:  & fett und groß & $\mathbf{A}$ \\
	\end{tabular}
	\vspace{3mm}\\
	\begin{tabular}{lll}
		Signale im Zeitbereich: & klein & $u(t)$ als Skalar bzw. $\mathbf{x}(t)$ als Vektor\\
		Signale im Frequenzbereich: & groß & $G(s)$ als Skalar bzw. $\mathbf{X}(s)$ als Vektor\\
	\end{tabular}
\end{table}
\\
Für zeitdiskrete Signale werden eckige statt runden Klammern verwendet. Folglich steht $y[k]$ für den $k$-ten Schritt des zeitdiskreten Signals $y$. Im Rahmen dieser Arbeit werden zeitdiskrete Schritte immer als äquidistant betrachtet.
Die Übertragungsfunktionen von Streckenmodellen sind mit $G(s)$ bzw. $G(z)$ und Regelungen mit $K(s)$ bzw $K(z)$ gekennzeichnet. Bei Signalen beschreibt $w$ eine Führungsgröße, $u$ eine Stellgröße $y$ eine Regelgröße und $z$ eine Störgröße. Der Buchstabe $e$ wird für die Regeldifferenz nach einem Summierglied verwendet. Der Übersichtlichkeit halber wird bei den kontinuierlichen Signalen auf die Nachführung des Zeitindex (t) verzichtet.